\chapter{De la Doctrina Cristiana y de sus partes principales.}

\section{LECCIÓN PRELIMINAR}

\pre{¿Sois cristiano?} Sí, señor; soy cristiano por la gracia de Dios. 

\pre{¿Por qué decís por la gracia de Dios?} Digo \textit{por la gracia de Dios} 
porque el ser cristiano es un don enteramente gratuito de Dios nuestro 
Señor, que no hemos podido merecer. 

\pre{¿Quién es verdadero cristiano?} Verdadero cristiano es el que está 
bautizado, cree y profesa la doctrina cristiana y obedece a los legítimos 
Pastores de la Iglesia. 

\pre{¿Qué es la doctrina cristiana?} Doctrina Cristiana es la doctrina que
nos enseñó Nuestro Señor Jesucristo para mostrarnos el camino de la 
salvación. 

\pre{¿Es necesario aprender la doctrina enseñada por Jesucristo?} Es 
necesario aprender la doctrina enseñada por Jesucristo, y faltan gravemente 
los que descuidan aprenderla. 

\pre{¿Tienen los padres y los amos obligación de mandar a sus hijos y 
dependientes al Catecismo?} Los padres y los amos tienen obligación de 
procurar que sus hijos y dependientes aprendan la doctrina cristiana, e 
incurren en culpa delante de Dios si descuidan esta obligación. 

\pre{¿De quién hemos de recibir y aprender la doctrina cristiana?} La
doctrina cristiana la hemos de recibir y aprender de la santa Iglesia Católica. 

\pre{¿Cómo estamos ciertos de que la doctrina cristiana que recibimos 
de la Santa Iglesia es realmente verdadera?} Estamos ciertos que la 
doctrina cristiana que recibimos de la Iglesia Católica es realmente verdadera 
porque Jesucristo, divino Autor de esta doctrina, la confió por medio de sus 
Apóstoles a la Iglesia fundada por El, a la cual constituyó Maestra infalible 
de todos los hombres y prometió su divina asistencia hasta el fin del mundo. 

\pre{¿Hay otras pruebas de la verdad de la doctrina cristiana?} La 
verdad de la doctrina cristiana se demuestra, además, por la santidad 
eminente de tantos que la profesaron y profesan, por la heroica fortaleza de 
los mártires, por su rápida y admirable propagación en el mundo y por su 
completa conservación por espacio de tantos siglos de varias y continuas 
luchas. 

\pre{¿Cuántas y cuáles son las partes principales y más necesarias de 
la doctrina cristiana?} Las partes principales y más necesarias de la 
doctrina cristiana son cuatro: \textit{El Credo, Padrenuestro, Mandamientos y 
Sacramentos}. 

\pre{¿Qué nos enseña el Credo?} El \textit{Credo} nos enseña los principales 
artículos de nuestra santa fe. 

\pre{¿Qué nos enseña el Padrenuestro?} El \textit{Padrenuestro} nos enseña todo 
lo que hemos de esperar de Dios y todo lo que hemos de pedirle. 

\pre{¿Qué nos enseñan los Mandamientos?} Los \textit{Mandamientos} nos 
enseña todo lo que hemos de hacer para agradar a Dios, que se resume en 
amar a Dios sobre todas las cosas y al prójimo como a nosotros mismos por 
amor de Dios. 

\pre{¿Qué nos enseña la doctrina de los Sacramentos?} La doctrina de 
los \textit{Sacramentos} nos enseña la naturaleza y buen uso de los medios instituidos por Jesucristo para perdonarnos los pecados, comunicarnos su gracia e 
infundir y acrecentar en nosotros las virtudes de la fe, de la esperanza y de la 
caridad. 
