\chapter{Del Símbolo de los Apóstoles llamado vulgarmente el «Credo»}
\section{DEL CREDO EN GENERAL}
\pre{¿Cuál es la primera parte de la doctrina cristiana?} La primera parte de la doctrina cristiana es el símbolo de los Apóstoles, llamado vulgarmente el CREDO.

\pre{¿Por qué llamáis al Credo SÍMBOLO DE LOS APÓSTOLES?} Llamo \textit{símbolo de los Apóstoles} al Credo porque es un compendio de las verdades de la fe enseñadas por los Apóstoles.

\pre{¿Cuántos son los artículos del CREDO?} Los artículos del Credo son doce.

\pre{Decidlos}

\begin{enumerate}
	\item Creo en Dios Padre Todopoderoso, Creador del cielo y de la tierra.
	\item Creo en Jesucristo, su único Hijo, nuestro Señor.
	\item Que fue concebido por obra y gracia del Espíritu Santo; nació de Santa María Virgen.
	\item Padeció bajo el poder de Poncio Pilato: fue crucificado, muerto y sepultado.
	\item Descendió a los infiernos: al tercer día resucitó de entre los muertos.
	\item Subió a los cielos: está sentado a la diestra de Dios Padre.
	\item Desde allí ha de venir a juzgar a los vivos y a los muertos.
	\item Creo en el Espíritu Santo.
	\item En la Santa Iglesia Católica: la Comunión de los Santos.
	\item El perdón de los pecados.
	\item La resurrección de los muertos.
	\item Y la vida eterna. Amén.
\end{enumerate}

\pre{¿Que quiere decirla palabra CREO?} La palabra Creo quiere decir: tengo por certísimo todo lo que en estos doce artículos se contiene, y lo creo con más firmeza que si lo viera con mis ojos, porque Dios, que ni puede engañarse ni engañarnos, lo ha revelado a la santa Iglesia Católica, y por medio de ella nos lo revela también a nosotros.

\pre{¿Que contienen los artículos del Credo?} Los artículos del Credo contienen todo lo que principalmente hemos de creer acerca de Dios, de Jesucristo y de la Iglesia.

\pre{¿Es muy bueno rezar a menudo el Credo?} Es provechosísimo rezar a menudo el Credo para grabar más y más en nuestro corazón las verdades de la fe.

\section{DEL PRIMER ARTÍCULO DEL SÍMBOLO}
\setcounter{sub}{1}
\subsection{De Dios Padre y de la Creación}

\pre{¿Que nos enseña el primer artículo: CREO EN DIOS PADRE TODOPODEROSO, CREADOR DEL CIELO Y DE LA TIERRA?} El primer artículo del Credo nos enseña que hay un solo Dios; que es todopoderoso, que creó el cielo y la tierra y todo lo que en el cielo y en la tierra se contiene.

\pre{¿Cómo sabemos que hay Dios?} Sabemos que hay Dios porque la razón lo demuestra y la fe lo confirma.

\pre{¿Por qué se dice que Dios es Padre?} Se dice que Dios es Padre: 1º. Porque es Padre, por naturaleza, de la segunda persona de la Santísima Trinidad, que es el Hijo engendrado por El. 2º. Porque Dios es Padre de todos los hombres que el ha creado, conserva y gobierna. 3º. Porque finalmente, es Padre por gracia de todos los buenos cristianos, que por eso se llaman hijos de Dios adoptivos.

\pre{¿Por qué el Padre es la Primera Persona de la Santísima Trinidad?} El Padre es la primera Persona de la Santísima Trinidad porque no procede de otra persona, sino que es el principio de las otras dos Personas, que son el Hijo y el Espíritu Santo.

\pre{¿Qué quiere decir TODOPODEROSO?} Todopoderoso quiere decir que Dios puede hacer todo cuanto quiere.

\pre{Dios no puede pecar ni morir ¿cómo, pues, se dice que todo lo puede?} Se dice que Dios todo lo puede, aunque no pueda pecar ni morir, porque el pecar o morir no es efecto de potencia, sino de flaqueza, la cual no puede hallarse en Dios, que es perfectísimo.

\pre{¿Qué quiere decir CREADOR DEL CIELO Y DE LA TIERRA?} Crear es hacer de nada algo; por esto se dice Creador del cielo y de la tierra, porque hizo de nada el cielo y la tierra y cuanto en el cielo y en la tierra se contiene.

\pre{¿Fue creado el mundo por el Padre solamente?} El mundo fue creado igualmente por las tres divinas Personas, porque todo cuanto hace una Persona respecto a las criaturas, lo hacen con el mismo acto las otras dos.

\pre{¿Por qué, pues, la creación se atribuye particularmente al Padre?} La creación se atribuye particularmente al Padre porque es efecto de la divina Omnipotencia; la cual se atribuye especialmente al Padre, como la sabiduría al Hijo y la bondad al Espíritu Santo, aunque las tres divinas Personas tienen la misma omnipotencia, sabiduría y bondad.

\pre{¿Tiene Dios cuidado del mundo y de todas las cosas que ha creado?} Si, señor; Dios tiene cuidado del mundo y de todas las cosas que ha creado, las conserva y gobierna con su infinita bondad y sabiduría, y nada sucede acá abajo sin que Dios lo quiera o permita.

\pre{¿Por qué decís que nada sucede sin que Dios lo quiera o lo permita?} Digo que nada sucede sin que Dios lo quiera o lo permita porque hay cosas que Dios quiere y manda y otras que no las impide, como es el pecado.

\pre{¿Por que Dios no impide el pecado?} Dios no impide el pecado porque aun del abuso, que el hombre hace de la libertad que El le dio, sabe sacar bien y hacer que brille más y más su misericordia o su justicia.


\subsection{De los Ángeles}

\pre{¿Cuáles son las criaturas más nobles que Dios ha creado?} Las criaturas
más nobles creadas por Dios son los Ángeles.

\pre{¿Quiénes son los Ángeles?} Los Ángeles son criaturas inteligentes y
puramente espirituales.

\pre{¿Para que fin creó Dios a los Ángeles?} Dios creó a los Ángeles para que
le honren y le sirvan y para hacerlos eternamente bienaventurados.

\pre{¿Qué forma o figura tienen los Ángeles?} Los Ángeles no tienen forma ni
figura alguna sensible, porque son puros espíritus, que subsisten sin necesidad de
estar unidos a cuerpo alguno.

\pre{¿Por qué, pues, se representan los Ángeles con formas sensibles?} Los
Ángeles se representan con formas sensibles: 1º, para ayudar a nuestra
imaginación; 2º, porque así han aparecido muchas veces a los hombres, como
leemos en las Santas Escrituras.

\pre{¿Permanecieron fieles a Dios todos los Ángeles?} No, señor; no
permanecieron fieles a Dios todos los Ángeles; antes, muchos de ellos, por
soberbia, pretendieron ser iguales a El e independientes, y por este pecado fueron
desterrados para siempre del paraíso y condenados al infierno.

\pre{¿Cómo se llaman los Ángeles desterrados para siempre del paraíso y
condenados al infierno?} Los Ángeles desterrados para siempre del paraíso y
condenados al infierno se llaman demonios, y su caudillo se llama Lucifer o
Satanás.

\pre{¿Pueden los demonios hacernos algún mal?} Sí, señor; los demonios
pueden hacernos mucho mal en el alma y en el cuerpo, si Dios les da licencia,
mayormente tentándonos a pecar.

\pre{¿Por qué nos tientan?} Los demonios nos tientan por la envidia que nos
tienen, la cual les hace desear nuestra eterna condenación, y por odio a Dios, cuya
imagen resplandece en nosotros.

\pre{¿Por qué permite Dios las tentaciones?} Dios permite las tentaciones
para que, venciéndolas con su gracia, ejercitemos las virtudes y adquiramos
merecimientos para el cielo.

\pre{¿Cómo se vencen las tentaciones?} Las tentaciones se vencen con la
vigilia, la oración y la mortificación cristiana.

\pre{¿Cómo se llaman los Ángeles que permanecieron fieles a Dios?} Los
Ángeles que permanecieron fieles a Dios se llaman Ángeles buenos, Espíritus
celestiales o simplemente Ángeles.

\pre{¿Qué fue de los Ángeles que permanecieron fieles a Dios?} Los Ángeles
que permanecieron fieles a Dios fueron confirmados en gracia, gozan para
siempre de la vista de Dios, le aman, le bendicen y le alaban eternamente.

\pre{¿Sírvese Dios de los Ángeles como de ministros suyos?} Sí, señor; Dios
se sirve de los Ángeles como de ministros suyos, y en especial a muchos de ellos
hace custodios y protectores nuestros.

\pre{¿Hemos de tener particular devoción al Ángel de nuestra Guarda?} Si,
señor; hemos de tener particular al Ángel de nuestra guarda, honrarle, implorar su
socorro, seguir sus inspiraciones y ser agradecidos a su continua asistencia.

\subsection{Del hombre}

\pre{¿Cuál es la criatura más noble que Dios ha puesto sobre la tierra?} La
criatura más noble que Dios ha puesto sobre la tierra es el hombre.

\pre{¿Qué es el hombre?} El hombre es una criatura racional compuesta de
alma y cuerpo.

\pre{¿Que es el alma?} El alma es la parte más noble del hombre, porque es
sustancia espiritual dotada de entendimiento y de voluntad, capaz de conocer a
Dios y de poseerle eternamente.

\pre{¿Puede verse y tocarse el alma humana?} El alma humana no puede
verse ni tocarse, porque es espíritu.

\pre{¿Muere con el cuerpo el alma humana?} El alma humana no muere
jamás; la fe y la misma razón prueban que es inmortal.

\pre{¿Es libre el hombre en sus acciones?} Sí, señor; el hombre es libre en sus
acciones, y todos nosotros sentimos dentro de nosotros mismos que podemos
hacer una cosa o no hacerla, o hacer una en vez de otra.

\pre{Explicadme con un ejemplo la libertad humana.} Al decir yo
voluntariamente una mentira, pienso que podía no decirla y callar, y que podía,
asimismo, hablar de otro modo, diciendo la verdad.

\pre{¿Por qué se dice que el hombre fue creado a imagen y semejanza de
Dios?} Se dice que el hombre fue creado a imagen y semejanza de Dios porque
el alma humana es espiritual y racional, libre en su obrar, capaz de conocer y amar
a Dios y gozarlo eternamente: perfecciones que son un reflejo de la infinita
grandeza del Señor.

\pre{¿En qué estado puso Dios a nuestros primeros padres, Adán y Eva?} Dios puso a Adán y a Eva en el estado de inocencia y gracia; más presto cayeron
de él por el pecado.

\pre{¿Dio el Señor otros dones a nuestros primeros padres, además de la
inocencia y de la gracia santificante?} Además de la inocencia y de la gracia
santificante, dio el Señor otros dones a nuestros primeros padres, que ellos debían
transmitir junto con la gracia santificante a sus descendientes, y eran: la integridad, o perfecta sujeción de la sensualidad de la razón; la inmortalidad; la inmunidad de todo dolor y miseria, y la ciencia proporcionada a su estado.

\pre{¿Cuál fue el pecado de Adán?} El pecado de Adán fue pecado de soberbia
y grave desobediencia.

\pre{¿Cuál fue el castigo de Adán y Eva?} Adán y Eva perdieron la gracia de
Dios y el derecho al cielo; fueron lanzados del paraíso terrenal, sujetos a muchas
miserias en el alma y en el cuerpo y condenados a morir.

\pre{Si Adán y Eva no hubiesen pecado, ¿hubieran estado exentos de la
muerte?} Si Adán y Eva no hubiesen pecado, tras una feliz estancia en este
mundo, hubieran sido trasladados por Dios al cielo, sin morir, para gozar una vida
eterna y gloriosa.

\pre{¿Eran estos dones debidos al hombre?} Estos dones no eran debidos al
hombre, sino absolutamente gratuitos y sobrenaturales, y por esto,
desobedeciendo Adán al divino mandamiento, pudo Dios, sin injusticia, privar de
ellos a Adán y a toda su posteridad.

\pre{¿Es este pecado únicamente propio de Adán?} Este pecado no es propio
únicamente de Adán, sino que también es nuestro, aunque de diverso modo. Es
propio de Adán porque él lo cometió con un acto de su voluntad, y por esto en él
fue personal. Es propio nuestro porque, habiendo pecado Adán en calidad de
cabeza y fuente de todo el linaje humano, viene transfundiéndose por natural
generación a todos sus descendientes, y por esto es para nosotros pecado original.

\pre{¿Cómo es posible que el pecado original se transfunda a todos los
hombres?} El pecado original se transfunde a todos los hombres porque,
habiendo conferido Dios al género humano en Adán la gracia santificante y los
otros dones sobrenaturales, a condición de que Adán no desobedeciese, habiendo
éste desobedecido, en su calidad de cabeza y padre de humano linaje, tornó la
naturaleza humana rebelde a Dios. Por esta causa, la naturaleza humana se
transfunde a todos los hombres descendientes de Adán en estado de rebelión a
Dios, privada de la gracia divina y de los otros dones.

\pre{¿Qué daños nos ha causado el pecado original?} Los daños que nos ha
causado el pecado original son la privación de la gracia, la pérdida de la
bienaventuranza, la ignorancia, la inclinación al mal, todas las miserias de esta vida y, en fin, la muerte.

\pre{¿Contraen todos los hombres el pecado original?} Si, señor; todos los
hombres contraen el pecado original, excepto la Santísima Virgen, que fue
preservada de Dios por singular privilegio, en previsión de los méritos de
Jesucristo Nuestro Salvador. Este privilegio se llama “la Inmaculada Concepción”
de María Santísima.

\pre{¿Podrían salvarse los hombres después del pecado de Adán?} Después
del pecado de Adán, los hombres no podían salvarse, a no usar Dios la
misericordia con ellos.

\pre{¿Cuál fue la misericordia que usó Dios con el linaje humano?} La
misericordia que usó Dios con el linaje humano fue prometer, desde luego, a
Adán el Redentor divino o Mesías, y enviarlo después a su tiempo para librar a los
hombres de la esclavitud del demonio y del pecado.

\pre{¿Quién es el Mesías prometido?} El Mesías prometido es Jesucristo,
como nos enseña el segundo artículo del Credo.

\section{DEL SEGUNDO ARTÍCULO}

\pre{¿Que nos enseña el segundo artículo : CREO EN JESUCRISTO, SU ÚNICO HIJO, NUESTRO SEÑOR?} El segundo artículo del Credo nos enseña que el Hijo de Dios es la segunda persona de la santísima Trinidad: que es Dios eterno, omnipotente, Creador y Señor como el Padre, que se hizo hombre para salvarnos, y que el Hijo de Dios hecho hombre se llama Jesucristo.

\pre{¿Por qué la segunda Persona se llama HIJO?} La segunda Persona se
llama Hijo porque es engendrada del Padre por vía de entendimiento desde toda
la eternidad, y por esto se llama también Verbo eterno del Padre.

\pre{Siendo también nosotros hijos de Dios ¿por qué Jesucristo se llama
HIJO ÚNICO DE DIOS PADRE?} Jesucristo se llama Hijo Único de Dios
Padre porque sólo El es el Hijo suyo por naturaleza, y nosotros somos hijos por
creación y adopción.

\pre{¿Por qué Jesucristo se llama NUESTRO SEÑOR?} Jesucristo se llama
Nuestro Señor porque además de habernos creado junto con el Padre y el
Espíritu Santo, en cuanto es Dios, nos ha redimido también en cuanto Dios y
hombre.

\pre{¿Por qué el Hijo de Dios hecho hombre se llama JESÚS?} El Hijo de
Dios hecho hombre se llama Jesús, que quiere decir Salvador, porque nos ha
salvado de la muerte eterna merecida por nuestros pecados.

\pre{¿Quién dio el nombre de JESÚS al Hijo de Dios hecho hombre?} El
nombre de Jesús lo dio al Hijo de Dios hecho hombre el mismo eterno Padre, por
medio del Arcángel San Gabriel, cuando éste anunció a la Virgen el misterio de la
Encarnación.

\pre{¿Por qué el Hijo de Dios hecho hombre se llama también CRISTO?}
El Hijo de Dios hecho hombre se llama también Cristo, que quiere decir ungido y
consagrado, porque antiguamente se ungían a los reyes, sacerdotes y profetas, y
Jesucristo es Rey de reyes, Sumo Sacerdote y Sumo Profeta.

\pre{¿Fue Jesucristo verdaderamente ungido y consagrado con unción corporal?} La unción de Jesucristo no fue corporal, como la de los antiguos reyes, sacerdotes y profetas, sino toda espiritual y divina, porque la plenitud de la
divinidad habita en El substancialmente.

\pre{¿Tuvieron los hombres algún conocimiento de Jesucristo antes de su
venida?} Si, señor; los hombres tuvieron conocimiento de Jesucristo antes de su
venida por la promesa del Mesías que hizo Dios a nuestros primeros padres, Adán
y Eva, y renovó a los Santos Patriarcas, y por las profecías y muchas figuras que le
señalaban.

\pre{¿Por dónde sabemos que Jesucristo es verdaderamente el Mesías y
Redentor prometido?} Sabemos que Jesucristo es verdaderamente el Mesías y
Redentor prometido por haberse cumplido en El: 1º, todo lo que anunciaban las
profecías; 2º, todo lo que representaban las figuras del Antiguo Testamento.

\pre{¿Qué predecían las profecías acerca del Redentor?} Las profecías
predecían la tribu y familia de la cual había de nacer el Redentor; el lugar y tiempo de su nacimiento; sus milagros y las más pequeñas circunstancias de su pasión y muerte; su resurrección y ascensión a los cielos; su reino espiritual, universal y perpetuo, que es la Santa Iglesia Católica.

\pre{¿Cuáles son las principales figuras del Redentor en el Antiguo
Testamento?} Las principales figuras del Redentor en el Antiguo Testamento
son el inocente Abel, el sumo sacerdote Melquisedech, el sacrificio de Isaac, José
vendido por sus hermanos, el profeta Jonás, el cordero pascual y la serpiente de
bronce levantada por Moisés en el desierto.

\pre{¿Cómo sabemos que Jesucristo es verdadero Dios?} Sabemos que
Jesucristo es verdadero Dios: 1º., por el testimonio del Padre cuando dijo: Este es
mi Hijo muy amado, en quien tengo todas mis complacencias, oídle. 2º., por la
atestación del mismo Jesucristo, confirmada con los milagros más estupendos. 3º.,
por la doctrina de los Apóstoles. 4º., por la tradición constante de la Iglesia
Católica.

\pre{¿Cuales son los principales milagros obrados por Jesucristo?} Los
principales milagros obrados por Jesucristo son, además de la resurrección, el
haber dado salud a los enfermos, vista a los ciegos, oído a los sordos, vida a los
muertos.

\section{DEL TERCER ARTÍCULO}

\pre{¿Que nos enseña el tercer artículo: QUE FUE CONCEBIDO POR
OBRA Y GRACIA DEL ESPÍRITU SANTO: NACIÓ DE SANTA MARÍA
VIRGEN?} El tercer artículo del Credo nos enseña que el Hijo de Dios tomó
cuerpo y alma, como tenemos nosotros, en las purísimas entrañas de María
Virgen, por obra del Espíritu Santo, y que nació de esta Virgen.

\pre{¿Concurrieron también el Padre y el Hijo a formar el cuerpo y crear el
alma de Jesucristo?} Si, señor; a formar el cuerpo y a crear el alma de Jesucristo
concurrieron las tres divinas personas.

\pre{¿Por qué se dice sólo: FUE CONCEBIDO POR OBRA Y GRACIA
DEL ESPÍRITU SANTO?} Se dice sólo: fue concebido por obra y gracia del
Espíritu Santo porque la Encarnación del Hijo de Dios fue obra de bondad y
amor, y las obras de bondad y amor se atribuyen al Espíritu Santo.

\pre{¿El Hijo de Dios, al hacerse hombre, ¿dejó de ser Dios?} No, señor; el
Hijo de Dios se hizo hombre sin dejar de ser Dios.

\pre{¿Luego Jesucristo es Dios y hombre juntamente?} Si, señor; el Hijo de
Dios encarnado, esto es, Jesucristo, es Dios y hombre juntamente, perfecto Dios
y perfecto hombre.

\pre{¿Luego en Jesucristo hay dos naturalezas?} Si, señor; en Jesucristo, que
es Dios y hombre, hay dos naturalezas: la divina y la humana.

\pre{¿Hay también en Jesucristo dos personas, la divina y la humana?} No,
señor; en el Hijo de Dios hecho hombre no hay más que una Persona, y ésta es la
divina.

\pre{¿Cuantas voluntades hay en Jesucristo?} En Jesucristo hay dos
voluntades: la una divina y la otra humana.

\pre{¿Tenía Jesucristo libre albedrío?} Si, señor; Jesucristo tenía libre albedrío,
más no podía obrar el mal, porque el poder obrar el mal es defecto, no perfección
de la libertad.

\pre{¿Son una misma Persona el Hijo de Dios y el Hijo de María?} El Hijo
de Dios y el Hijo de María son una misma Persona, esto es, Jesucristo, verdadero
Dios y verdadero hombre.

\pre{¿Es la Virgen María Madre de Dios?} Si, señor; la Virgen María es Madre
de Dios, porque es Madre de Jesucristo, que es verdadero Dios.

\pre{¿De qué manera vino a ser María Madre de Jesucristo?} María vino a
ser Madre de Jesucristo únicamente por obra y gracia del Espíritu Santo.

\pre{¿Es de fe que María fue siempre Virgen?} Si, señor; es de fe que María
Santísima fue siempre Virgen, y es llamada la Virgen por excelencia.

\textbf{96 bis. ¿Es de fe que María está en cuerpo y alma en el Cielo?} Si, señor;
desde el 1º de noviembre de 1950, es dogma de fe que María Santísima terminado
el curso de su mortal vida, fue llevada en cuerpo y alma a los Cielos. Este
privilegio se llama “la Asunción de María”.

\section{DEL CUARTO ARTÍCULO}

\pre{¿Qué nos enseña el artículo cuarto: PADECIÓ BAJO EL PODER DE
PONCIO PILATO: FUE CRUCIFICADO, MUERTO Y SEPULTADO?}
En cuarto artículo del Credo nos enseña que Jesucristo, para redimir al mundo
con su sangre preciosa, padeció bajo Poncio Pilato, murió en la Cruz y fue
sepultado.

\pre{¿Que expresa la palabra padeció?} 
La palabra padeció expresa todas las penas que Jesucristo sufrió en su pasión.

\pre{¿Murió Jesucristo en cuanto Dios o en cuanto hombre?} Jesucristo
murió en cuanto hombre, porque en cuanto Dios no podía padecer ni morir.

\pre{¿Qué especie de suplicio era el de la cruz?} El suplicio de la cruz era el
más cruel y afrentoso de todos los suplicios.

\pre{¿Quién fue el que condenó a Jesucristo a ser crucificado?} El que
condenó a Jesucristo a ser crucificado fue Poncio Pilato, gobernador de la Judea,
quien había reconocido la inocencia del Salvador, más cedió vilmente a las
amenazas del pueblo de Jerusalén.

\pre{¿No hubiera podido Jesucristo librarse de las manos de los judíos y
de Pilato?} Si, señor; Jesucristo hubiera podido librarse de las manos de los
judíos y de Pilato, más se sujetó voluntariamente a padecer y morir para salvarnos,
por saber que así lo quería su eterno Padre, y aún salió al encuentro de sus
enemigos y se dejó espontáneamente prender y llevar a la muerte.

\pre{¿Dónde fue crucificado Jesucristo?} Jesucristo fue crucificado en el
monte Calvario.

\pre{¿Qué hizo Jesucristo en la Cruz?} Jesucristo en la Cruz rogó por sus
enemigos; dio su misma Madre, María Santísima, por madre a su discípulo San
Juan, y en él a todos nosotros; ofreció su muerte en sacrificio y satisfizo a la
justicia de Dios por los pecados de los hombres.

\pre{¿No bastara que viniese un Ángel para satisfacer por nosotros?} No,
señor; no bastara que viniese un Ángel a satisfacer por nosotros, porque la ofensa
hecha a Dios por el pecado era, en cierta manera, infinita, y para satisfacer por ella se requería una persona que tuviese un mérito infinito.

\pre{¿Era menester que Jesucristo fuese Dios y hombre juntamente para
satisfacer a la divina justicia?} Si, señor; era menester que Jesucristo fuese
hombre para que pudiese padecer y morir, y que fuese Dios para que sus
padecimientos fuesen de valor infinito.

\pre{¿Por qué era necesario que los méritos de Jesucristo fuesen de valor
infinito?} Era necesario que los méritos de Jesucristo fuesen de valor infinito
porque la majestad de Dios, ofendida por el pecado, es infinita.

\pre{¿Era necesario que Jesucristo padeciese tanto?} No, señor; no era
absolutamente necesario que Jesús padeciese tanto, porque el menor de sus
padecimientos hubiera sido suficiente para nuestra redención, siendo cualquiera
acción suya de valor infinito.

\pre{¿Por qué, pues, quiso Jesús padecer tanto?} Quiso Jesús padecer tanto
para satisfacer más copiosamente a la divina justicia, para mostrarnos más su
amor y para inspirarnos sumo horror al pecado.

\pre{¿Sucedieron algunos prodigios a la muerte de Jesús?} Si, señor; a la
muerte de Jesús se oscureció el sol, se estremeció la tierra, abriéndose los
sepulcros y muchos muertos resucitaron.

\pre{¿Dónde fue sepultado el cuerpo de Jesucristo?} El cuerpo de Jesucristo
fue sepultado en un sepulcro nuevo, cavado en la peña del monte, no lejos del
lugar donde le habían crucificado.

\pre{¿Se separó del cuerpo y del alma la divinidad en la muerte de
Jesucristo?} En la muerte de Jesucristo, la divinidad no se separó ni del cuerpo
ni del alma, sino solamente el alma se separó del cuerpo.

\pre{¿Por quién murió Jesucristo?} Jesucristo murió por la salvación de todos
los hombres y por todos ellos satisfizo.

\pre{Si Jesucristo murió por todos los hombres, ¿por qué no todos se salvan?} Jesucristo murió por todos; pero no todos se salvan, porque o no le
quieren reconocer o no guardan su ley, o no se valen de los medios de
santificación que nos dejó.

\pre{¿Basta para salvarnos que Jesucristo haya muerto por nosotros?} Para
salvarnos no basta que Jesucristo haya muerto por nosotros, sino que es necesario
aplicar a cada uno el fruto y los méritos de su pasión y muerte, lo que se hace
principalmente por medio de los sacramentos instituidos a este fin por el mismo
Jesucristo, y como muchos no reciben los sacramentos, o no los reciben bien, por
esto hacen para sí mismos inútil la muerte de Jesucristo.

\section{DEL QUINTO ARTÍCULO}

\pre{¿Qué nos enseña el quinto artículo: DESCENDIÓ A LOS INFIERNOS: AL TERCER DÍA RESUCITÓ DE ENTRE LOS MUERTOS?} El quinto artículo del Credo nos enseña: que el alma de Jesucristo, separada ya del cuerpo, fue al Limbo de los Santos Padres y que al tercer día se unió de nuevo a su cuerpo para no separarse jamás.

\pre{¿Qué se entiende aquí por Infierno?} Por infierno se entiende aquí el
Limbo de los Santos Padres, es decir, el lugar donde las almas de los justos eran
recogidas y esperaban la redención de Jesucristo.

\pre{¿Por qué las almas de los Santos Padres no fueron introducidas en el
cielo antes de la muerte de Jesucristo?} Las almas de los Santos Padres no
fueron introducidas en el cielo antes de la muerte de Jesucristo porque por el
pecado de Adán el cielo estaba cerrado, y convenía que el primero que entrase en
él fuese Jesucristo, que con su muerte lo abrió de nuevo.

\pre{¿Por qué Jesucristo quiso dilatar hasta el tercer día su propia
resurrección?} Jesucristo quiso dilatar hasta el tercer día su propia resurrección
para mostrar con evidencia que verdaderamente había muerto.

\pre{¿Fue la resurrección de Jesucristo semejante a la resurrección de los
otros hombres resucitados?} No, señor; la resurrección de Jesucristo no fue
semejante a la resurrección de los otros hombres resucitados, porque Jesucristo
resucitó por su propia virtud, y los demás fueron resucitados por la virtud de
Dios.

\section{DEL SEXTO ARTICULO}

\pre{¿Qué nos enseña el sexto artículo: SUBIÓ A LOS CIELOS: ESTÁ
SENTADO A LA DIESTRA DE DIOS PADRE?} El sexto artículo del
Credo nos enseña que Jesucristo, cuarenta días después de su resurrección, subió
por sí mismo al cielo en presencia de sus discípulos, y que, siendo como Dios
igual al Padre en la gloria, fue como hombre ensalzado sobre todos los Ángeles y
Santos y constituido Señor de todas las cosas.

\pre{¿Por qué Jesucristo después de su resurrección se quedó cuarenta
días en la tierra antes de subir al cielo?} Jesucristo, después de su
resurrección, quedóse cuarenta días en la tierra, antes de subir al cielo, para probar con varias apariciones que verdaderamente había resucitado, y para instruir más y más y conformar a los Apóstoles en las verdades de la fe.

\pre{¿Por qué subió Jesucristo al cielo?} Jesucristo subió al cielo: 

\begin{enumerate}
	\item Para tomar posesión de su reino, conquistado con su muerte.
	\item Para prepararnos tronos de gloria y para ser nuestro Medianero y Abogado cerca del Padre.
	\item Para enviar el Espíritu Santo a sus Apóstoles.
\end{enumerate}

\pre{¿Por qué se dice de Jesucristo que subió a los cielos y de su Madre
Santísima que fue asunta?} Dícese de Jesucristo que subió a los cielos y de su
Madre Santísima que fue asunta, porque Jesucristo, por ser Hombre-Dios, subió
al cielo por su propia virtud, pero su Madre, como era criatura, aunque la más
digna de todas, subió al cielo por la virtud de Dios.

\pre{Explicadme las palabras: ESTÁ SENTADO A LA DIESTRA DE DIOS
PADRE.} La palabra está sentado significa la eterna y pacífica posesión que
Jesucristo tiene de su gloria, y la expresión a la diestra de Dios Padre quiere decir que ocupa el puesto de honor sobre todas las criaturas.

\section{DEL SÉPTIMO ARTÍCULO}

\pre{¿Qué nos enseña el séptimo artículo: DESDE ALLÍ HA DE VENIR
A JUZGAR A LOS VIVOS Y A LOS MUERTOS?} El séptimo artículo del
Credo nos enseña que al fin del mundo Jesucristo, lleno de gloria y majestad,
vendrá del cielo para juzgar a todos los hombres, buenos y malos, y dar a cada
uno el premio o el castigo que hubiere merecido.

\pre{Si todos, inmediatamente después de la muerte, hemos de ser
juzgados por Jesucristo en el juicio particular, ¿por qué todos hemos de ser
juzgados en el juicio universal?} Hemos de ser juzgados todos en el juicio
universal por varias razones:

\begin{enumerate}
	\item Para gloria de Dios.
	\item Para gloria de Jesucristo.
	\item Para gloria de los Santos.
	\item Para confusión de los malos.
	\item Para que el cuerpo tenga con el alma su sentencia de premio o de castigo.
\end{enumerate}

\pre{¿Cómo se manifestará la gloria de Dios en el juicio universal?} En el
juicio universal se manifestará la gloria de Dios, porque todos conocerán con
cuanta justicia gobierna Dios el mundo, aunque ahora se ven muchas veces
afligidos los buenos y en prosperidad los malos.

\pre{¿Cómo se manifestará en el juicio universal la gloria de Jesucristo?}
En el juicio universal se manifestará la gloria de Jesucristo porque habiendo sido
injustamente condenado por los hombres, aparecerá entonces a la faz de todo el
mundo como juez supremo de todos.

\pre{¿Cómo se manifestará la gloria de los Santos en el juicio universal?}
En el juicio universal se manifestará la gloria de los Santos porque muchos de
ellos, que murieron despreciados de los malos, serán glorificados a la vista de todo
el mundo.

\pre{¿Cuál será en el juicio universal la confusión de los malos?} En el
juicio universal será grandísima la confusión de los malos, mayormente la de
aquellos que oprimieron a los justos o procuraron en vida ser estimados como
hombres buenos y virtuosos, al ver descubiertos a todo el mundo los pecados que
cometieron, aún los más secretos.

\section{DEL OCTAVO ARTÍCULO}

\pre{¿Qué nos enseña el octavo artículo: CREO EN EL ESPÍRITU
SANTO?} El octavo artículo del Credo nos enseña que hay Espíritu Santo,
tercera Persona de la Santísima Trinidad, que es Dios eterno, infinito,
omnipotente, Criador y Señor de todas las cosas, como el Padre y el Hijo.

\pre{¿De quién procede el Espíritu Santo?} El Espíritu Santo procede del
Padre y del Hijo, por vía de voluntad y de amor, como de un solo principio.

\pre{Si el Hijo procede del Padre, y el Espíritu Santo procede del Padre y
del Hijo, parece que el Padre y el Hijo sean antes que el Espíritu Santo,
¿cómo, pues, se dice que todas tres Personas son eternas?} Se dice que
todas tres Personas son eternas porque el Padre desde todas la eternidad engendra
al Hijo, y del Padre y del Hijo procede desde toda la eternidad el Espíritu Santo.

\pre{¿Por qué la tercera Persona de la Santísima Trinidad se llama
particularmente con el nombre de Espíritu Santo?} La tercera Persona de la
Santísima Trinidad se llama particularmente con el nombre de Espíritu Santo
porque procede del Padre y del Hijo por vía de aspiración y de amor.

\pre{¿Qué obra se atribuye especialmente al espíritu Santo?} Al Espíritu
Santo se atribuye especialmente la santificación de las almas.

\pre{¿No nos santifican el Padre y el Hijo lo mismo que el Espíritu Santo?}
Si, señor; todas tres personas nos santifican igualmente.

\pre{Pues, ¿por qué la santificación de las almas se atribuye en particular
al Espíritu Santo?} La santificación de las almas se atribuye en particular al
Espíritu Santo porque es obra de amor, y las obras de amor se atribuyen al
Espíritu Santo.

\pre{¿Cuándo bajó el Espíritu Santo sobre los Apóstoles?} El Espíritu Santo
bajó sobre los Apóstoles el día de Pentecostés; es decir, cincuenta días después de
la Resurrección de Jesucristo y diez después de su Ascensión.

\pre{¿Dónde estaban los Apóstoles los diez días antes de Pentecostés?}
Los Apóstoles estaban reunidos en el Cenáculo en compañía de la Virgen María y
de otros discípulos, y perseveraban en oración esperando al Espíritu Santo que
Jesucristo les había prometido.

\pre{¿Qué efectos produjo el espíritu Santo en los Apóstoles?} El Espíritu
Santo confirmó en la fe a los Apóstoles, los llenó de luz, de fortaleza, de caridad y de la abundancia de todos sus dones.

\pre{¿Fue el Espíritu Santo enviado para sólo los Apóstoles?} El Espíritu
Santo fue enviado para toda la Iglesia y para todas las almas fieles.

\pre{¿Que obra el Espíritu Santo en la Iglesia?} El Espíritu Santo, como el
alma en el cuerpo, vivifica con su gracia y dones a la Iglesia, establece en ella el
reinado de la verdad y del amor y la asiste para que lleve con seguridad a sus hijos
por el camino del cielo.

\section{DEL NOVENO ARTÍCULO}
\setcounter{sub}{1}
\subsection{De la Iglesia en general}

\pre{¿Qué nos enseña el noveno artículo: EN LA SANTA IGLESIA
CATÓLICA: LA COMUNIÓN DE LOS SANTOS?} El noveno artículo del
Credo nos enseña que Jesucristo fundó en la tierra una sociedad visible, que se
llama Iglesia Católica, y que todos los que forman parte de esta Iglesia están en
comunión entre sí.

\pre{¿Por qué después del artículo que trata del Espíritu Santo se habla
inmediatamente de la Iglesia Católica?} Después del artículo que trata del
Espíritu Santo se habla inmediatamente de la Iglesia Católica, para indicar que
toda la santidad de la misma Iglesia se deriva del Espíritu Santo, que es el autor de toda santidad.

\pre{¿Que quiere decir esta palabra Iglesia?} La palabra Iglesia quiere decir
convocación o reunión de muchas personas.

\pre{¿Quién nos ha convocado o llamado a la Iglesia de Jesucristo?} Dios,
por una gracia particular, nos ha llamado a la Iglesia de Jesucristo, para que con la luz de la fe y la observancia de la divina ley le demos el debido culto y lleguemos a la vida eterna.

\pre{¿Dónde se hallan los miembros de la Iglesia?} Los miembros de la
Iglesia se hallan, parte en el cielo, y forman la Iglesia triunfante; parte en el
purgatorio, y forman la Iglesia purgante o paciente, y parte sobre la tierra, y
forman la Iglesia militante.

\pre{¿Constituyen una sola Iglesia estas diversas partes de la Iglesia?} Si,
señor; estas diversas partes de la Iglesia constituyen una misma Iglesia y un solo
cuerpo, porque tienen una misma cabeza, que es Jesucristo; un mismo espíritu,
que las anima y une entre sí, un mismo fin, que es la bienaventuranza eterna, la
cual unos miembros gozan ya y otros la aguardan.

\pre{¿A qué parte de la Iglesia se refiere principalmente este noveno
artículo del Credo?} Este noveno artículo del Credo se refiere principalmente a
la Iglesia militante, que es la Iglesia en que estamos los presentes.

\subsection{De la Iglesia en particular}

\pre{¿Qué es la Iglesia Católica?} La Iglesia Católica es la sociedad o
congregación de todos los bautizados que, viviendo en la tierra, profesan la misma
fe y ley de Cristo, participan en los mismos Sacramentos y obedecen a los
legítimos Pastores, principalmente al Romano Pontífice.

\pre{Decid distintamente: ¿qué es necesario para ser miembro de la
Iglesia?} Para ser miembro de la Iglesia es necesario estar bautizado, creer y
profesar la doctrina de Jesucristo, participar de los mismos Sacramentos,
reconocer al Papa y a los otros Pastores legítimos de la Iglesia.

\pre{¿Quienes son los Pastores legítimos de la Iglesia?} Los Pastores
legítimos de la Iglesia son el Romano Pontífice, o sea, el Papa, que es el Pastor
universal, y los Obispos. Además, con dependencia de los Obispos y del Papa,
tienen parte en el oficio de Pastores los otros sacerdotes, y en especial los
párrocos.

\pre{¿Por qué decís que el Romano Pontífice es el Pastor universal de la
Iglesia?} Porque Jesucristo dijo a San Pedro, primer Papa: “Tú eres Pedro, y
sobre esta piedra edificaré mi Iglesia, y te daré las llaves del reino de los cielos, y todo lo que atares en la tierra será atado en el cielo, y lo que desatares en la tierra, será desatado también en el cielo.” Y, asimismo, le dijo: “Apacienta mis corderos, apacienta mis ovejas.”

\pre{¿No pertenecen, pues, a la Iglesia de Jesucristo tantas sociedades de
hombres bautizados que no reconocen al Romano Pontífice por cabeza?}
No, señor; todos los que no reconocen al Romano Pontífice por cabeza no
pertenecen a la Iglesia de Jesucristo.

\pre{¿Cómo puede distinguirse la Iglesia de Jesucristo de tantas
sociedades o sectas fundadas por los hombres y que se dicen cristianas?}
Entre tantas sociedades o sectas fundadas por los hombres, que se dicen
cristianas, puédese fácilmente distinguir la verdadera Iglesia de Jesucristo por
cuatro notas, porque sólo ella es UNA, SANTA, CATÓLICA y APOSTÓLICA.

\pre{¿Por qué la Iglesia verdadera es UNA?} La Iglesia verdadera es UNA
porque sus hijos, de cualquier tiempo y lugar, están unidos entre sí en una misma
fe, un mismo culto, una misma ley y en la participación de unos sacramentos bajo
una misma cabeza visible, el Romano Pontífice.

\pre{¿No podría haber más Iglesias?} No, señor; no puede haber más Iglesias,
porque así como no hay más que un solo Dios, una Fe y un solo Bautismo, así no
hay ni puede haber más que una sola y verdadera Iglesia.

\pre{¿Pero no se llaman también Iglesias los fieles unidos de una nación o
diócesis?} Se llaman también Iglesias los fieles unidos de una nación o diócesis,
pero con todo eso no son sino partes de la Iglesia universal, con la que forman
una sola Iglesia.

\pre{¿Por qué la Iglesia verdadera es SANTA?} La Iglesia verdadera es
SANTA porque santa es su cabeza invisible, que es Jesucristo, santos muchos de
sus miembros, santas su fe, su ley, sus sacramentos, y fuera de ella no hay ni
puede haber verdadera santidad.

\pre{¿Por qué la Iglesia verdadera es CATÓLICA?} La Iglesia verdadera es
CATÓLICA que quiere decir universal, porque abraza los fieles de todos los
tiempos y lugares, de toda edad y condición, y todos los hombres del mundo son
llamados a formar parte de ella.

\pre{¿Por qué la Iglesia verdadera es, además, APOSTÓLICA?} La Iglesia
verdadera es, además, APOSTÓLICA porque se remonta sin interrupción hasta
los Apóstoles; porque cree y enseña todo lo que ellos creyeron y enseñaron y
porque es guiada y gobernada por los Pastores que legítimamente les suceden.

\pre{¿Y por qué la Iglesia verdadera se llama, asimismo, ROMANA?} La
Iglesia verdadera se llama, asimismo, ROMANA porque los cuatro caracteres de
unidad, santidad, catolicidad y apostolicidad se hallan sólo en la Iglesia que
reconoce por cabeza al Obispo de Roma, sucesor de San Pedro.

\pre{¿Cómo está constituida la Iglesia de Jesucristo?} La Iglesia de
Jesucristo está constituida como una verdadera y perfecta sociedad, y en ella,
como en toda persona moral, podemos distinguir alma y cuerpo.

\pre{¿En que consiste el alma de la Iglesia?} El alma de la Iglesia consiste en
lo que tiene de interno y espiritual, que es la fe, la esperanza y la caridad, los dones de la gracia y del Espíritu Santo y todos los celestiales tesoros que le provienen de los merecimientos de Cristo Redentor y de los Santos.

\pre{¿En qué consiste el cuerpo de la Iglesia?} El cuerpo de la iglesia cosiste
en lo que tiene de visible y externo, ya en la asociación de los congregados, ya en
el culto y ministerio de la enseñanza, ya en su orden exterior y gobierno.

\pre{¿Basta para salvarse ser como quiera miembro de la Iglesia Católica?}
No, señor; no basta para salvarse ser como quiera miembro de la Iglesia Católica,
sino que es necesario ser miembro vivo.

\pre{¿Cuáles son los miembros vivos de la Iglesia?} Los miembros vivos de
la Iglesia son todos y solamente los justos; a saber, los que están actualmente en
gracia de Dios.

\pre{¿Y cuáles son los miembros muertos?} Miembros muertos de la Iglesia
son los fieles que se hallan en pecado mortal.

\pre{¿Puede alguien salvarse fuera de la Iglesia Católica, Apostólica,
Romana?} No, señor; fuera de la Iglesia Católica, Apostólica y Romana, nadie
puede salvarse, como nadie pudo salvarse del diluvio fuera del Arca de Noé, que
era figura de esta Iglesia.

\pre{¿Cómo, pues, se salvaron los antiguos Patriarcas y Profetas y todos los
otros justos del Antiguo Testamento?} Todos los justos del Antiguo
Testamento se salvaron en virtud de la fe que tenían en Cristo futuro, mediante la
cual ya pertenecían espiritualmente a esta Iglesia.

\pre{¿Podría salvarse quien sin culpa se hallase fuera de la Iglesia?} Quién
sin culpa, es decir, de buena fe, se hallase fuera de la Iglesia y hubiese recibido el bautismo o, a lo menos, tuviese el deseo implícito de recibirlo y buscase, además, sinceramente la verdad y cumpliese la voluntad de Dios lo mejor que pudiese, este tal, aunque separado del cuerpo de la Iglesia, estaría unido al alma de ella y, por consiguiente, en camino de salvación.

\pre{¿Se salvaría quien, siendo miembro de la Iglesia Católica, no
practicase sus enseñanzas?} Quien, siendo miembro de la Iglesia Católica, no
practicase sus enseñanzas, sería miembro muerto y, por tanto, no se salvaría, pues
para la salvación de un adulto se requiere no sólo el bautismo y la fe, sino también
obras conformes a la fe.

\pre{¿Estamos obligados a creer todas las verdades que la Iglesia nos
enseña?} Si, señor; estamos obligados a creer todas las verdades que la Iglesia
nos enseña, y Jesucristo declara que el que no cree, ya está condenado.

\pre{¿Estamos, además, obligados a cumplir todo lo que la Iglesia nos
manda?} Si, señor; estamos obligados a cumplir todo lo que la Iglesia nos
manda, porque Jesucristo ha dicho a los Pastores de la Iglesia: “El que a vosotros
oye, a Mí me oye, y el que a vosotros desprecia, a Mí me desprecia”.

\pre{¿Puede errar la Iglesia en los que nos propone para creer?} No, señor;
en las cosas que nos propone para creer la Iglesia no puede errar, porque, según la
promesa de Jesucristo, está perennemente asistida por el Espíritu Santo.

\pre{¿Es, pues, infalible la Iglesia Católica?} Si, señor; la Iglesia católica es
infalible, y a esta causa, los que rechazan sus definiciones pierden la fe y se hacen herejes.

\pre{¿Puede la Iglesia Católica ser destruida o perecer?} No, señor; la Iglesia
Católica puede ser perseguida, pero no destruida ni perecer. Durará hasta el fin
del mundo, porque hasta el fin del mundo estará con ella Jesucristo, como El lo
ha prometido.

\pre{¿Por qué es tan perseguida la Iglesia Católica?} La Iglesia Católica es
tan perseguida porque también fue perseguido su divino Fundador y porque
reprueba los vicios, combate pasiones y condena todas las injusticias y errores.

\pre{¿Tienen los católicos otros deberes que cumplir con la Iglesia?} Todo
católico ha de profesar un amor sin límites a la Iglesia, estimarse por infinitamente honrado y feliz de pertenecer a ella, y procurar su gloria y acrecentamiento por cuantos medios pueda.

\subsection{De la Iglesia docente y de la Iglesia discente o enseñada}

\pre{¿Hay alguna distinción entre los miembros que componen la Iglesia?}
Entre los miembros que componen la Iglesia hay una distinción notabilísima,
porque hay en ella quien manda y quien obedece, quien enseña y quien es
enseñado.

\pre{¿Cómo se llama la parte de la Iglesia que enseña?} La parte de la Iglesia
que enseña se llama docente o enseñante.

\pre{¿Cómo se llama la parte de la Iglesia que aprende?} La parte de la
Iglesia que aprende se llama discente o enseñada.

\pre{¿Quien ha establecido está distinción en la Iglesia?} Esta distinción en
la Iglesia la ha establecido el mismo Jesucristo.

\pre{¿Son, pues, dos Iglesias distintas la IGLESIA DOCENTE y la
IGLESIA DISCENTE?} La Iglesia docente y la Iglesia discente son partes
distintas de una misma y única Iglesia, como en el cuerpo humano la cabeza es
distinta a los otros miembros, y con todo forma con ellos un solo cuerpo.

\pre{¿Quienes componen la Iglesia DOCENTE?} Componen la Iglesia
docente todos los Obispos, con el Romano Pontífice a la cabeza, ya se hallen
dispersos, ya congregados en Concilio.

\pre{¿Y quienes componen la Iglesia DISCENTE o enseñada?} Componen
la Iglesia discente o enseñada todos los fieles.

\pre{¿Quiénes, pues, tienen en la Iglesia la autoridad de enseñar?} La
autoridad de enseñar la tienen en la Iglesia el Papa y los Obispos, y con
dependencia de ellos, los demás sagrados Ministros.

\pre{¿Estamos obligados a escuchar a la IGLESIA DOCENTE?} Si, por
cierto; todos estamos obligados a escuchar a la Iglesia docente, so pena de eterna
condenación, porque Jesucristo dijo a los Pastores de la Iglesia en la persona de
los Apóstoles: “El que a vosotros oye, a Mí me oye, y el que a vosotros desprecia,
a Mí me desprecia”.

\pre{¿Tiene la Iglesia algún otro poder además de la autoridad de enseñar?}
Si, señor; además de la autoridad de enseñar, tiene la Iglesia especialmente el
poder de administrar las cosas santas, hacer leyes y exigir su cumplimiento.

\pre{¿Viene del pueblo el poder que tienen los miembros de la Jerarquía
eclesiástica?} El poder que tienen los miembros de la Jerarquía eclesiástica no
viene del pueblo, y decir esto sería herejía, sino que viene únicamente de Dios.

\pre{¿A quién compete el ejercicio de estos poderes?} El ejercicio de estos
poderes compete exclusivamente al orden jerárquico, es decir, al Papa y a los
Obispos a él subordinados.

\subsection{Del Papa y de los Obispos}

\pre{¿Quién es el Papa?} El Papa, a quien llamamos asimismo Sumo Pontífice
o también Romano Pontífice, es el sucesor de San Pedro en la Cátedra de Roma,
Vicario de Jesucristo y cabeza visible de la Iglesia.

\pre{¿Por qué el Romano Pontífice es sucesor de San Pedro?} El Romano
Pontífice es sucesor de San Pedro porque San Pedro unió en su persona la
dignidad de Obispo de Roma y de cabeza dela Iglesia; estableció en Roma por
divina disposición su sede, y allí murió; por esto, el que es elegido Obispo de
Roma, es también heredero de toda su autoridad.

\pre{¿Por qué el Romano Pontífice es Vicario de Jesucristo?} El Romano
Pontífice es Vicario de Jesucristo porque le representa en la tierra y hace sus veces en el gobierno de la Iglesia.

\pre{¿Porqué el Romano Pontífice es cabeza visible de la Iglesia?} El
Romano Pontífice es cabeza visible de la Iglesia porque él la rige visiblemente con
la misma autoridad de Jesucristo, que es cabeza invisible.

\pre{¿Qué dignidad es, pues, la del Papa?} La dignidad del Papa es la mayor
entre todas las dignidades de la tierra, con que ejerce supremo e inmediato poder
sobre todos y cada uno de los Pastores y de los fieles.

\pre{¿Puede errar el Papa al enseñar a la Iglesia?} El Papa no puede errar, es
decir, es infalible en las definiciones que atañen a la fe y a las costumbres.

\pre{¿Por qué motivo el Papa es infalible?} El Papa es infalible por la
promesa de Jesucristo y por la continua asistencia del Espíritu Santo.

\pre{¿Cuándo es infalible el Papa?} El Papa es infalible sólo cuando, en calidad
de Pastor y Maestro de todos los cristianos, en virtud de su suprema y apostólica
autoridad, define que una doctrina acerca de la fe o de las costumbres debe ser
abrazada por la Iglesia universal.

\pre{¿Qué pecado cometería el que no creyese las solemnes definiciones
del Papa?} El que no creyese las solemnes definiciones del Papa, o aunque sólo
dudase de ellas, pecaría contra la fe, y si persistiese obstinadamente en esa
incredulidad, ya no sería católico, sino hereje.

\pre{¿A qué fin ha otorgado Dios al Papa el don de la infalibilidad?} Dios
ha otorgado al Papa el don de la infalibilidad para que todos estemos ciertos y
seguros de la verdad que la Iglesia nos enseña.

\pre{¿Cuándo definió la Iglesia que el Papa es infalible?} La Iglesia definió
en el Concilio Vaticano I que el Papa es infalible, y si alguien presumiese
contradecir a esta definición, sería hereje y excomulgado.

\pre{¿Ha establecido la Iglesia alguna nueva verdad de fe al definir que el
Papa es infalible?} No, señor; la Iglesia no ha establecido ninguna nueva verdad
de fe al definir que el Papa es infalible, sino solamente ha definido, para oponerse
a los nuevos errores, que la infalibilidad del Papa, contenida ya en la Sagrada
Escritura y en la Tradición, es una verdad revelada por Dios, y, por consiguiente,
que ha de creerse como dogma o artículo de fe.

\pre{¿Cómo debe portarse todo católico respecto al Papa?} Todo católico
debe reconocer al Papa como Padre, Pastor y Maestro universal, y estar unido con
él de entendimiento y corazón.

\pre{¿Quienes son por institución divina los personajes más venerados de
la Iglesia después del Papa?} Los personajes más venerados de la Iglesia
después del Papa son, por institución divina, los Obispos.

\pre{¿Quienes son los Obispos?} Los Obispos son los Pastores de los fieles,
puestos por el Espíritu Santo para gobernar la Iglesia de Dios en las sedes que se
les han encomendado, con dependencia del Romano Pontífice.

\pre{¿Qué es el Obispo en su propia diócesis?} El Obispo en su propia
diócesis es el Pastor legítimo, el Padre, el Maestro, el superior de todos los fieles, eclesiásticos y seglares, que pertenecen a la misma diócesis.

\pre{¿Por qué llamamos al Obispo Pastor legítimo?} Llamamos al Obispo
Pastor legítimo porque la jurisdicción, esto es, el poder que tiene de gobernar a los fieles de la propia diócesis, se le ha conferido según las normas y leyes de la
Iglesia.

\pre{¿De quienes son sucesores el Papa y los Obispos?} El Papa es sucesor
de San Pedro, Príncipe de los Apóstoles, y los Obispos son sucesores de los
Apóstoles en lo que mira al gobierno ordinario de la Iglesia.

\pre{¿Debe el fiel estar unido a su propio Obispo?} Sí, señor; todo fiel,
eclesiástico o seglar, debe estar unido de entendimiento y de corazón a su propio
Obispo, en gracia y comunión con la Sede Apostólica.

\pre{¿Cómo debe portarse todo fiel con su propio Obispo?} Todo fiel,
eclesiástico o seglar, debe reverenciar, amar y honrar a su Obispo y prestarle
obediencia en todo lo que se refiere a la cura de almas y al gobierno espiritual de
la diócesis.

\pre{¿De quién se ayuda el Obispo en la cura de almas?} El Obispo, en la
cura de almas, se ayuda de los sacerdotes, y principalmente de los párrocos.

\pre{¿Quién es el párroco?} El Párroco es un sacerdote designado para presidir
y dirigir con dependencia del Obispo, una parte de la diócesis, que se llama
parroquia.

\pre{¿Cuáles son los deberes de los fieles para con su párroco?} Los fieles
deben estar unidos con su párroco, escucharle con docilidad y profesarle respeto y
sumisión en todo lo que atañe al régimen de la Parroquia.

\subsection{De la Comunión de los Santos}

\pre{¿Qué nos enseña el noveno artículo del Credo con aquellas palabras:
LA COMUNIÓN DE LOS SANTOS?} Con las palabras: La comunión de los
Santos, el noveno artículo del Credo nos enseña que en la Iglesia, por la íntima
unión que existe entre todos sus miembros, son comunes los bienes espirituales
que le pertenecen, así internos como externos.

\pre{¿Cuáles son en la Iglesia los bienes comunes internos?} Los bienes
comunes internos en la Iglesia son: la gracia que se recibe en los Sacramentos, la
fe, la esperanza, la caridad, los méritos infinitos de Jesucristo, los merecimientos
sobreabundantes de la Virgen y de los Santos y el fruto de todas las buenas obras
que se hacen en la misma Iglesia.

\pre{¿Cuáles son los bienes comunes externos en la Iglesia?} Los bienes
externos comunes en la Iglesia son: los Sacramentos, el Santo Sacrificio de la
Misa, las públicas oraciones, las funciones religiosas y las demás prácticas
exteriores que unen a los fieles entre sí.

\pre{¿Entran todos los hijos de la Iglesia en esta comunión de bienes?} En
la comunión de los bienes internos entran los cristianos que están en gracia de
Dios; pero los que están en pecado mortal no participan de estos bienes.

\pre{¿Por qué no participan de estos bienes los que están en pecado
mortal?} Porque la gracia de Dios es la que junta a los fieles con Dios y entre sí;
por esto, lo que están en pecado mortal, como no tienen la gracia de Dios, son
excluidos de la comunión de los bienes espirituales.

\pre{Luego, ¿no perciben ninguna utilidad de los bienes internos y
espirituales de la Iglesia los cristianos que están en pecado mortal?} Los
cristianos que están en pecado mortal no dejan de percibir alguna utilidad de los
bienes internos y espirituales de la Iglesia de que están privados, en cuanto
conservan el carácter de cristiano, que es indeleble, y son ayudados de las
oraciones y buenas obras de los fieles para alcanzar la gracia de convertirse a Dios.

\pre{¿Pueden participar de los bienes externos de la Iglesia los que están
en pecado mortal?} Los que están en pecado mortal pueden participar de los
bienes externos de la Iglesia, con tal que no estén separados de la Iglesia por la
excomunión.

\pre{¿Por qué los miembros de esta comunión, tomados en conjunto, se
llaman santos?} Los miembros de esta comunión se llaman santos, porque todos
son llamados a la santidad y fueron santificados por medio del Bautismo, y
muchos de ellos han llegado ya a la perfecta santidad.

\pre{¿Se extiende también al cielo y al purgatorio la comunión de los
santos?} Si, señor; la comunión de los santos se extiende también al cielo y al
purgatorio, porque la caridad une las tres Iglesias: triunfante, purgante y militante; los santos ruegan a Dios por nosotros y por las almas del purgatorio, y nosotros damos honor y gloria a los santos, y podemos aliviar a las almas del purgatorio aplicándoles en sufragio misas, limosnas, indulgencias y otras buenas obras.

\subsection{De los que están fuera de la Iglesia}

\pre{¿Quiénes son los que no pertenecen a la comunión de los Santos?}
No pertenecen a la comunión de los santos en la otra vida los condenados, y en
ésta, los que están fuera de la verdadera Iglesia.

\pre{¿Quiénes están fuera de la verdadera Iglesia?} Está fuera de la
verdadera Iglesia los infieles, los judíos, los herejes, los apóstatas, los cismáticos y los excomulgados.

\pre{¿Quiénes son los infieles?} Infieles son los que no tienen el Bautismo ni
creen en Jesucristo, o porque creen y adoran falsas divinidades, cómo los
idólatras, o porque, aun admitiendo al único verdadero Dios, no creen en Cristo
Mesías, ni como venido ya en la persona de Jesucristo ni como que ha de venir:
tales son los mahometanos y otros semejantes.

\pre{¿Quiénes son los judíos?} Judíos son los que profesan la ley de Moisés,
no han recibido el Bautismo y no creen en Jesucristo.

\pre{¿Quiénes son los herejes?} Herejes son los bautizados que rehusan con
pertinacia creer alguna verdad revelada por Dios y enseñada como de fe por la
Iglesia Católica; por ejemplo los arrianos, los nestorianos y las varias sectas de los protestantes.

\pre{¿Quiénes son los apóstatas?} Apóstatas son los que abjuran, esto es,
niegan con acto externo la fe católica que antes profesaban.

\pre{¿Quiénes son los cismáticos?} Cismáticos son los cristianos que, sin
negar explícitamente ningún dogma, se separan voluntariamente de la Iglesia de
Jesucristo, esto es, de sus legítimos Pastores.

\pre{¿Quiénes son los excomulgados?} Los excomulgados son aquellos que
por faltas gravísimas son castigados por el Papa o por el Obispo con la pena de
excomunión, en cuya virtud son, como indignos, separados del cuerpo de la
Iglesia, que espera y desea su conversión.

\pre{¿Débese temer la excomunión?} La excomunión debe temer
grandemente, porque es la pena más grave y más terrible que puede imponer la
Iglesia a sus hijos rebeldes y obstinados.

\pre{¿De qué bienes quedan privados los excomulgados?} Los
excomulgados quedan privados de las oraciones públicas, de los sacramentos, de
las indulgencias y, después de sentencia condenatoria o declaratoria, también de
sepultura eclesiástica.

\pre{¿Podemos ayudar en alguna manera a los excomulgados?} Podemos
ayudar en alguna manera a los excomulgados y a todos los que están fuera de la
Iglesia con saludables avisos, con oraciones y buenas obras, suplicando al Señor
que por su misericordia les otorgue la gracia de convertirse a la fe y entrar en la
comunión de los Santos.

\section{DEL DÉCIMO ARTÍCULO}

\pre{¿Qué nos enseña el décimo artículo: EL PERDÓN DE LOS
PECADOS?} El décimo artículo del Credo nos enseña que Jesucristo ha dejado
a su Iglesia el poder de perdonar los pecados.

\pre{¿Puede la Iglesia perdonar toda clase de pecados?} Si, señor; la Iglesia
puede perdonar todos los pecados, por muchos y graves que sean, porque
Jesucristo le ha dado plena potestad para atar y desatar.

\pre{¿Quiénes son los que en la Iglesia ejercen esta potestad de perdonar
los pecados?} Los que en la Iglesia ejercen la potestad de perdonar los pecados
son, en primer lugar, el Papa, que es el único que posee la plenitud de esta
potestad; luego los Obispos y, con dependencia de los Obispos, los sacerdotes.

\pre{¿Cómo perdona la Iglesia los pecados?} La Iglesia perdona los pecados
por los méritos de Jesucristo, confiriendo los sacramentos instituidos por El con
este fin, principalmente el Bautismo y la Penitencia.

\section{DEL UNDÉCIMO ARTÍCULO}

\pre{¿Qué nos enseña el undécimo artículo: LA RESURRECCIÓN DE
LOS MUERTOS?} El undécimo artículo del Credo nos enseña que todos los
hombres resucitarán, volviendo a tomar cada alma el cuerpo que tuvo en esta
vida.

\pre{¿Cómo sucederá la resurrección de los muertos?} La resurrección de los
muertos sucederá por la virtud de Dios omnipotente, a quien nada es imposible.

\pre{¿Cuándo acaecerá la resurrección de los muertos?} La resurrección de
los muertos acaecerá al fin del mundo, y entonces seguirá el juicio universal.

\pre{¿Por qué ha dispuesto Dios la resurrección de los cuerpos?} Dios ha
dispuesto la resurrección de los cuerpos para que, habiendo el alma obrado el bien
o el mal junto con el cuerpo, sea también junto con el cuerpo premiada o
castigada.

\pre{¿Resucitarán todos los hombres de la misma manera?} No, señor; sino
que habrá grandísima diferencia entre los cuerpos de los escogidos y los cuerpos
de los condenados, porque sólo los cuerpos de los escogidos tendrán, a semejanza
de Jesucristo resucitado, las dotes de los cuerpos gloriosos.

\pre{¿Cuáles son las dotes que adornarán los cuerpos de los escogidos?}
Las dotes que adornarán los cuerpos gloriosos de los escogidos son: 

\begin{enumerate}
	\item La impasibilidad, por la que no podrán ya estar sujetos a males y dolores de ningún 	género, ni a la necesidad de comer, descansar o de otra cosa.
	\item La claridad, con la que brillarán como el sol y como otras tantas estrellas.
	\item La agilidad, con que 	podrán trasladarse en un momento y sin fatiga de un lugar a otro, y de la tierra al cielo.
	\item La sutileza, con que sin obstáculo alguno podrán penetrar cualquier
	cuerpo, como lo hizo Jesucristo resucitado.
\end{enumerate}

\pre{¿Cómo serán los cuerpos de los condenados?} Los cuerpos de los
condenados estarán privados de las dotes de los cuerpos gloriosos y llevarán la
horrible marca de su eterna condenación.

\section{DEL DUODÉCIMO ARTÍCULO}

\pre{¿Qué nos enseña el último artículo: Y LA VIDA ETERNA?} El último
artículo del Credo nos enseña que, después de la vida presente, hay otra, o
eternamente bienaventurada para los escogidos en el cielo o eternamente infeliz
para los condenados al infierno.

\pre{¿Podemos comprender la bienaventuranza del cielo?} No, señor; no
podemos comprender la bienaventuranza de la gloria, porque sobrepuja nuestro
limitado entendimiento y porque los bienes del cielo no pueden compararse con
los bienes de este mundo.

\pre{¿En qué consiste la bienaventuranza de los escogidos?} La
bienaventuranza de los escogidos consiste en ver, amar y poseer por siempre a
Dios, fuente de todo bien.

\pre{¿En qué consiste la infelicidad de los condenados?} La infelicidad de
los condenados consiste en ser privados por siempre de la vista de Dios y
castigados con eternos tormentos en el infierno.

\pre{¿Son únicamente para las almas los bienes del cielo y los males del
infierno?} Los bienes del cielo y los males del infierno son ahora únicamente
para las almas, porque solamente las almas está ahora en el cielo o en el infierno;
pero después de la resurrección, los hombres serán o felices o atormentados para
siempre en alma y cuerpo.

\pre{¿Serán iguales para los bienaventurados los bienes del cielo y para los
condenados los males del infierno?} Los bienes del cielo para los
bienaventurados y los males de infierno para los condenados serán iguales en la
sustancia y en la duración eterna; más en la medida o en los grados serán mayores
o menores, según los méritos o deméritos de cada cual.

\pre{¿Que quiere decir la palabra AMÉN al final del Credo?} La palabra
Amén al fin de las oraciones significa: Así sea; al fin del Credo significa: Así es,
que vale tanto como decir: Creo que es la pura verdad cuanto en estos doce
artículos se contiene y estoy más cierto de ello que si lo viese con mis propios
ojos.